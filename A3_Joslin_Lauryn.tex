% Options for packages loaded elsewhere
\PassOptionsToPackage{unicode}{hyperref}
\PassOptionsToPackage{hyphens}{url}
%
\documentclass[
]{article}
\usepackage{lmodern}
\usepackage{amssymb,amsmath}
\usepackage{ifxetex,ifluatex}
\ifnum 0\ifxetex 1\fi\ifluatex 1\fi=0 % if pdftex
  \usepackage[T1]{fontenc}
  \usepackage[utf8]{inputenc}
  \usepackage{textcomp} % provide euro and other symbols
\else % if luatex or xetex
  \usepackage{unicode-math}
  \defaultfontfeatures{Scale=MatchLowercase}
  \defaultfontfeatures[\rmfamily]{Ligatures=TeX,Scale=1}
\fi
% Use upquote if available, for straight quotes in verbatim environments
\IfFileExists{upquote.sty}{\usepackage{upquote}}{}
\IfFileExists{microtype.sty}{% use microtype if available
  \usepackage[]{microtype}
  \UseMicrotypeSet[protrusion]{basicmath} % disable protrusion for tt fonts
}{}
\makeatletter
\@ifundefined{KOMAClassName}{% if non-KOMA class
  \IfFileExists{parskip.sty}{%
    \usepackage{parskip}
  }{% else
    \setlength{\parindent}{0pt}
    \setlength{\parskip}{6pt plus 2pt minus 1pt}}
}{% if KOMA class
  \KOMAoptions{parskip=half}}
\makeatother
\usepackage{xcolor}
\IfFileExists{xurl.sty}{\usepackage{xurl}}{} % add URL line breaks if available
\IfFileExists{bookmark.sty}{\usepackage{bookmark}}{\usepackage{hyperref}}
\hypersetup{
  pdftitle={A3\_Joslin\_Lauryn\_},
  pdfauthor={Lauryn Joslin},
  hidelinks,
  pdfcreator={LaTeX via pandoc}}
\urlstyle{same} % disable monospaced font for URLs
\usepackage[margin=1in]{geometry}
\usepackage{color}
\usepackage{fancyvrb}
\newcommand{\VerbBar}{|}
\newcommand{\VERB}{\Verb[commandchars=\\\{\}]}
\DefineVerbatimEnvironment{Highlighting}{Verbatim}{commandchars=\\\{\}}
% Add ',fontsize=\small' for more characters per line
\usepackage{framed}
\definecolor{shadecolor}{RGB}{248,248,248}
\newenvironment{Shaded}{\begin{snugshade}}{\end{snugshade}}
\newcommand{\AlertTok}[1]{\textcolor[rgb]{0.94,0.16,0.16}{#1}}
\newcommand{\AnnotationTok}[1]{\textcolor[rgb]{0.56,0.35,0.01}{\textbf{\textit{#1}}}}
\newcommand{\AttributeTok}[1]{\textcolor[rgb]{0.77,0.63,0.00}{#1}}
\newcommand{\BaseNTok}[1]{\textcolor[rgb]{0.00,0.00,0.81}{#1}}
\newcommand{\BuiltInTok}[1]{#1}
\newcommand{\CharTok}[1]{\textcolor[rgb]{0.31,0.60,0.02}{#1}}
\newcommand{\CommentTok}[1]{\textcolor[rgb]{0.56,0.35,0.01}{\textit{#1}}}
\newcommand{\CommentVarTok}[1]{\textcolor[rgb]{0.56,0.35,0.01}{\textbf{\textit{#1}}}}
\newcommand{\ConstantTok}[1]{\textcolor[rgb]{0.00,0.00,0.00}{#1}}
\newcommand{\ControlFlowTok}[1]{\textcolor[rgb]{0.13,0.29,0.53}{\textbf{#1}}}
\newcommand{\DataTypeTok}[1]{\textcolor[rgb]{0.13,0.29,0.53}{#1}}
\newcommand{\DecValTok}[1]{\textcolor[rgb]{0.00,0.00,0.81}{#1}}
\newcommand{\DocumentationTok}[1]{\textcolor[rgb]{0.56,0.35,0.01}{\textbf{\textit{#1}}}}
\newcommand{\ErrorTok}[1]{\textcolor[rgb]{0.64,0.00,0.00}{\textbf{#1}}}
\newcommand{\ExtensionTok}[1]{#1}
\newcommand{\FloatTok}[1]{\textcolor[rgb]{0.00,0.00,0.81}{#1}}
\newcommand{\FunctionTok}[1]{\textcolor[rgb]{0.00,0.00,0.00}{#1}}
\newcommand{\ImportTok}[1]{#1}
\newcommand{\InformationTok}[1]{\textcolor[rgb]{0.56,0.35,0.01}{\textbf{\textit{#1}}}}
\newcommand{\KeywordTok}[1]{\textcolor[rgb]{0.13,0.29,0.53}{\textbf{#1}}}
\newcommand{\NormalTok}[1]{#1}
\newcommand{\OperatorTok}[1]{\textcolor[rgb]{0.81,0.36,0.00}{\textbf{#1}}}
\newcommand{\OtherTok}[1]{\textcolor[rgb]{0.56,0.35,0.01}{#1}}
\newcommand{\PreprocessorTok}[1]{\textcolor[rgb]{0.56,0.35,0.01}{\textit{#1}}}
\newcommand{\RegionMarkerTok}[1]{#1}
\newcommand{\SpecialCharTok}[1]{\textcolor[rgb]{0.00,0.00,0.00}{#1}}
\newcommand{\SpecialStringTok}[1]{\textcolor[rgb]{0.31,0.60,0.02}{#1}}
\newcommand{\StringTok}[1]{\textcolor[rgb]{0.31,0.60,0.02}{#1}}
\newcommand{\VariableTok}[1]{\textcolor[rgb]{0.00,0.00,0.00}{#1}}
\newcommand{\VerbatimStringTok}[1]{\textcolor[rgb]{0.31,0.60,0.02}{#1}}
\newcommand{\WarningTok}[1]{\textcolor[rgb]{0.56,0.35,0.01}{\textbf{\textit{#1}}}}
\usepackage{graphicx,grffile}
\makeatletter
\def\maxwidth{\ifdim\Gin@nat@width>\linewidth\linewidth\else\Gin@nat@width\fi}
\def\maxheight{\ifdim\Gin@nat@height>\textheight\textheight\else\Gin@nat@height\fi}
\makeatother
% Scale images if necessary, so that they will not overflow the page
% margins by default, and it is still possible to overwrite the defaults
% using explicit options in \includegraphics[width, height, ...]{}
\setkeys{Gin}{width=\maxwidth,height=\maxheight,keepaspectratio}
% Set default figure placement to htbp
\makeatletter
\def\fps@figure{htbp}
\makeatother
\setlength{\emergencystretch}{3em} % prevent overfull lines
\providecommand{\tightlist}{%
  \setlength{\itemsep}{0pt}\setlength{\parskip}{0pt}}
\setcounter{secnumdepth}{-\maxdimen} % remove section numbering

\title{A3\_Joslin\_Lauryn\_}
\author{Lauryn Joslin}
\date{26/01/2022}

\begin{document}
\maketitle

\hypertarget{r-markdown}{%
\subsection{R Markdown}\label{r-markdown}}

\begin{Shaded}
\begin{Highlighting}[]
\KeywordTok{setwd}\NormalTok{(}\StringTok{"/Users/LaurynJoslin/Desktop/A3_Joslin_Lauryn_/InputData"}\NormalTok{)}
\NormalTok{dat <-}\StringTok{ }\KeywordTok{read.csv}\NormalTok{(}\StringTok{"FallopiaData.csv"}\NormalTok{)}
\end{Highlighting}
\end{Shaded}

\begin{Shaded}
\begin{Highlighting}[]
\KeywordTok{library}\NormalTok{(dplyr)}
\end{Highlighting}
\end{Shaded}

\begin{verbatim}
## 
## Attaching package: 'dplyr'
\end{verbatim}

\begin{verbatim}
## The following objects are masked from 'package:stats':
## 
##     filter, lag
\end{verbatim}

\begin{verbatim}
## The following objects are masked from 'package:base':
## 
##     intersect, setdiff, setequal, union
\end{verbatim}

\begin{Shaded}
\begin{Highlighting}[]
\KeywordTok{library}\NormalTok{(rmarkdown)}
\end{Highlighting}
\end{Shaded}

\begin{Shaded}
\begin{Highlighting}[]
\CommentTok{#Remove rows with ‘Total’ biomass < 60}
\NormalTok{dat <-}\StringTok{ }\KeywordTok{filter}\NormalTok{(dat, Total }\OperatorTok{<}\StringTok{ }\DecValTok{60}\NormalTok{)}
\end{Highlighting}
\end{Shaded}

\begin{Shaded}
\begin{Highlighting}[]
\CommentTok{#Reorder the columns so that they are in the order: ‘Total’, ‘Taxon’, ‘Senario’, ‘Nutrients’, and remove the other columns}
\NormalTok{dat <-}\StringTok{ }\KeywordTok{select}\NormalTok{(dat, Total, Taxon, Scenario, Nutrients)}
\KeywordTok{head}\NormalTok{(dat)}
\end{Highlighting}
\end{Shaded}

\begin{verbatim}
##   Total Taxon Scenario Nutrients
## 1 50.22 japon      low       low
## 2 41.71 japon      low       low
## 3 41.81 japon      low       low
## 4 48.27 japon      low       low
## 5 55.42 japon      low       low
## 6 42.68 japon      low       low
\end{verbatim}

\begin{Shaded}
\begin{Highlighting}[]
\CommentTok{#Make a new column TotalG, which converts the ‘Total’ column from mg to grams AND replace Total with TotalG, and add it to the dataset.}
\NormalTok{dat}\OperatorTok{$}\NormalTok{TotalG <-}\StringTok{ }\NormalTok{dat}\OperatorTok{$}\NormalTok{Total}\OperatorTok{/}\DecValTok{1000}
\NormalTok{dat <-}\StringTok{ }\KeywordTok{select}\NormalTok{(dat, TotalG, Taxon, Scenario, Nutrients)}
\end{Highlighting}
\end{Shaded}

\begin{Shaded}
\begin{Highlighting}[]
\CommentTok{#Write a custom function that will take two inputs from the user: 1. a vector of data to #process (e.g. column from a data.frame object) and 2. a string that defines what calculation to perform.}
\CommentTok{#if string #2 is "Average" then calculate the average value for the column named in vector #1}
\CommentTok{#if string #2 is "Sum" then calculate the sum of values for the column named in vector #1}
\CommentTok{#if string #2 is "Observations" then count the number of observed values for the column named in vector #1}
\CommentTok{#if string #2 is anything else, then output an error to the user }


\NormalTok{myfun <-}\StringTok{ }\ControlFlowTok{function}\NormalTok{(}\DataTypeTok{var1=}\NormalTok{dat}\OperatorTok{$}\NormalTok{Scenario, }\DataTypeTok{var2=}\NormalTok{ xyz)\{}
  \ControlFlowTok{if}\NormalTok{ (var2 }\OperatorTok{==}\StringTok{ }\NormalTok{Average)\{}
\NormalTok{  Average <-}\StringTok{ }\KeywordTok{Mean}\NormalTok{(var1)}
\NormalTok{  \}}
  \ControlFlowTok{if}\NormalTok{ (var2 }\OperatorTok{==}\StringTok{ }\NormalTok{Sum)\{}
\NormalTok{  Sum <-}\StringTok{ }\KeywordTok{sum}\NormalTok{(var1)}
\NormalTok{  \}}
  \ControlFlowTok{if}\NormalTok{ (var2 }\OperatorTok{==}\StringTok{ }\NormalTok{Observations)\{}
\NormalTok{  Observations <-}\StringTok{ }\KeywordTok{count}\NormalTok{(var1)}
\NormalTok{  \}}
  \ControlFlowTok{else}\NormalTok{ \{}
    \KeywordTok{print}\NormalTok{(}\StringTok{"Error"}\NormalTok{)}
\NormalTok{  \}}
\NormalTok{\}}
\CommentTok{#Write some R code that uses your function to count the total number of observations in the 'Taxon' column.}

\NormalTok{myfun <-}\StringTok{ }\ControlFlowTok{function}\NormalTok{(}\DataTypeTok{var1=}\NormalTok{dat}\OperatorTok{$}\NormalTok{Taxon, }\DataTypeTok{var2=}\NormalTok{Observations)\{}
  \ControlFlowTok{if}\NormalTok{ (var2 }\OperatorTok{==}\StringTok{ }\NormalTok{Average)\{}
\NormalTok{  Average <-}\StringTok{ }\KeywordTok{Mean}\NormalTok{(var1)}
\NormalTok{  \}}
  \ControlFlowTok{if}\NormalTok{ (var2 }\OperatorTok{==}\StringTok{ }\NormalTok{Sum)\{}
\NormalTok{  Sum <-}\StringTok{ }\KeywordTok{sum}\NormalTok{(var1)}
\NormalTok{  \}}
  \ControlFlowTok{if}\NormalTok{ (var2 }\OperatorTok{==}\StringTok{ }\NormalTok{Observations)\{}
\NormalTok{  Observations <-}\StringTok{ }\KeywordTok{count}\NormalTok{(var1)}
\NormalTok{  \}}
  \ControlFlowTok{else}\NormalTok{ \{}
    \KeywordTok{print}\NormalTok{(}\StringTok{"Error"}\NormalTok{)}
\NormalTok{  \}}
\NormalTok{\}}

\CommentTok{#Write some R code that uses your function  to calculate the average TotalG for each of the two Nutrient concentrations}

\NormalTok{myfun <-}\StringTok{ }\ControlFlowTok{function}\NormalTok{(}\DataTypeTok{var1=}\NormalTok{dat}\OperatorTok{$}\NormalTok{TotalG, }\DataTypeTok{var2=}\NormalTok{Average)\{}
  \ControlFlowTok{if}\NormalTok{ (var2 }\OperatorTok{==}\StringTok{ }\NormalTok{Average)\{}
\NormalTok{  Average <-}\StringTok{ }\KeywordTok{Mean}\NormalTok{(var1)}
\NormalTok{  \}}
  \ControlFlowTok{if}\NormalTok{ (var2 }\OperatorTok{==}\StringTok{ }\NormalTok{Sum)\{}
\NormalTok{  Sum <-}\StringTok{ }\KeywordTok{sum}\NormalTok{(var1)}
\NormalTok{  \}}
  \ControlFlowTok{if}\NormalTok{ (var2 }\OperatorTok{==}\StringTok{ }\NormalTok{Observations)\{}
\NormalTok{  Observations <-}\StringTok{ }\KeywordTok{count}\NormalTok{(var1)}
\NormalTok{  \}}
  \ControlFlowTok{else}\NormalTok{ \{}
    \KeywordTok{print}\NormalTok{(}\StringTok{"Error"}\NormalTok{)}
\NormalTok{  \}}
\NormalTok{\}}
\end{Highlighting}
\end{Shaded}

\begin{Shaded}
\begin{Highlighting}[]
\CommentTok{#Write (i.e. save) the new data to a file called "WrangledData.csv" in the Output folder.}
\NormalTok{dat <-}\StringTok{ "WrangledData.csv"}
\CommentTok{#write.table(WrangledData.csv, file = "/Users/LaurynJoslin/Desktop/A3_Joslin_Lauryn_/Output")}
\end{Highlighting}
\end{Shaded}

\end{document}
